% 下列中文名詞的定義,如果以註解方式關閉取消,
% 則會以系統原先的預設值 (英文) 替代
% 名詞 \prechaptername 預設值為 Chapter
% 名詞 \postchaptername 預設值為空字串
% 名詞 \tablename 預設值為 Table
% 名詞 \figurename 預設值為 Figure

% \renewcommand\prechaptername{第} % 出現在每一章的開頭的「第 x 章」
% \renewcommand\postchaptername{章}
% \renewcommand{\tablename}{表} % 在文章中 table caption 會以「表 x」表示
% \renewcommand{\figurename}{圖} % 在文章中 figure caption 會以「圖 x」表示

% 下列中文名詞的定義,用於論文固定的各部分之命名 (出現於目錄與該頁標題)
\newcommand{\nameInnerCover}{教授推薦書}
\newcommand{\nameCommitteeForm}{論文口試委員審定書}
\newcommand{\nameCopyrightForm}{授權書}
\newcommand{\nameCabstract}{中文摘要}
\newcommand{\nameEabstract}{英文摘要}
\newcommand{\nameAckn}{誌謝}
\newcommand{\nameToc}{目錄}
\newcommand{\nameLot}{表目錄}
\newcommand{\nameTof}{圖目錄}
\newcommand{\nameToa}{演算法目錄}
\renewcommand{\listalgorithmcfname}{演算法目錄}
\newcommand{\nameSlist}{符號說明}
\newcommand{\nameRef}{參考文獻}
\newcommand{\nameVita}{自傳}

% -- Chapter style
\usepackage{zhnumber}

\titleformat{\chapter}{\centering\normalfont\huge\bfseries}{第\zhnumber{\thechapter} 章}{\chapterfontsize}{\Huge}