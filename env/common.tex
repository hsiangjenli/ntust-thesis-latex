% 除非校方修改了論文格式 (margins, header, footer, 浮水印)
% 或者需要增加所用的 LaTeX 套件,
% 或者要改預設中文字型、編碼
% 否則毋須修改本檔內容
% 論文撰寫,請修改以 my_  開頭檔名的各檔案
\usepackage{draftwatermark}
\usepackage{subcaption}
% \usepackage{subfig}
\usepackage{titlesec}
\usepackage{float}
\usepackage{titlecaps}
% \usepackage[nospace]{cite}  % for smart citation
\usepackage{geometry}  % for easy margin settings
%\usepackage[dvipdfm]{graphicx}  % for graphic   using eps
\usepackage{graphicx}  % for graphic   using eps
\usepackage[graphicx]{realboxes}

%\usepackage{epstopdf} % 當使用pdflatex時打開,如使用latex則不需開啟,此功能為將xxx.eps 自動判讀為XXX.pdf

\usepackage{algpseudocodex}
\usepackage{amsmath}
\usepackage[ruled,vlined,linesnumbered,commentsnumbered,longend]{algorithm2e} 

% margins setting
\geometry{verbose,a4paper,tmargin=3.5cm,bmargin=2cm,lmargin=3cm,rmargin=3cm}

\usepackage{amsmath} % 各式 AMS 數學功能
\usepackage{mathrsfs} %草寫體數學符號,在數學模式裡用 \mathscr{E} 得草寫 E
\usepackage{listings} % 程式列表套件

% listing setting
\lstset{breaklines=true,% 過長的程式行可斷行
extendedchars=false,% 中文處理不需要 extendedchars
texcl=true,% 中文註解需要有 TeX 處理過的 comment line, 所以設成 true
comment=[l]\%\%,% 以雙「百分號」做為程式中文註解的起頭標記,配合 MATLAB
basicstyle=\small,% 小號字體, 約 10 pt 大小
commentstyle=\upshape,% 預設是斜體字,會影響註解裏的英文,改用正體
%language=Octave % 會將一些 octave 指令以粗體顯示
}

\usepackage{url} % 在文稿中引用網址,可以用 \url{http://www.ntust.edu.tw} 方式

% 插圖套件 graphicx
% 使用者工作流程是用 pdftex 還是 latex + dvipdfmx?
% 視情況而有不同的參數
% 這裡作自動判斷
% 參考自
% http://www.tex.ac.uk/cgi-bin/texfaq2html?label=ifpdf
%\newcommand\mydvipdfmxflow{dvipdfmx}
%\newcommand\mypdftexflow{pdftex}
%
%\ifx\pdfoutput\undefined
%  % not running pdftex
%  \usepackage[dvipdfm]{graphicx}
%  \newcommand\myworkflow{dvipdfmx}  % set the flag for hyperref
%\else
%  \ifx\pdfoutput\relax
%    % not running pdftex
%    \usepackage[dvipdfm]{graphicx}
%    \newcommand\myworkflow{dvipdfmx}  % set the flag
%  \else
%    % running pdftex, with...
%    \ifnum\pdfoutput>0
%      % ... PDF output
%      \usepackage[pdftex]{graphicx}
%      \newcommand\myworkflow{pdftex}  % set the flag
%    \else
%      %...DVI output
%      \usepackage[dvipdfm]{graphicx}
%      \newcommand\myworkflow{dvipdfmx}  % set the flag
%    \fi
%  \fi
%\fi

\usepackage{fancyhdr}  % 借用增強功能型 header 套件來擺放浮水印 
% (佔用了 central header)
% 不需要浮水印的使用者仍可利用此套件,產生所需的 header, footer
%
% 啟動 fancy header/footer 套件
\pagestyle{fancy}
\fancyhead{}  % reset left, central, right header to empty
\fancyfoot[C]{\thepage} %中間 footer 擺放頁碼
\renewcommand{\headrulewidth}{0pt} % header 的直線; 0pt 則無線

% global page layout
\newcommand{\mybaselinestretch}{1.5}  %行距 1.5 倍 + 20%, (約為 double space)
\renewcommand{\baselinestretch}{\mybaselinestretch}  % 論文行距預設值
\parskip=2ex  % 段落之間的間隔為兩個 x 的高度
\parindent = 24Pt  % 段首內縮由 CJK 控制,所以這裡就設成不內縮

% watermark
\SetWatermarkText{}
\SetWatermarkAngle{0}
\SetWatermarkScale{2.5}

% -- 中文粗體 --
% value > 0
\def\xeCJKembold{0.4}

% hack into xeCJK, you don't need to understand it
\def\saveCJKnode{\dimen255\lastkern}
\def\restoreCJKnode{\kern-\dimen255\kern\dimen255}

% save old definition of \CJKsymbol and \CJKpunctsymbol for CJK output
\let\CJKoldsymbol\CJKsymbol
\let\CJKoldpunctsymbol\CJKpunctsymbol

% apply pdf literal fake bold
\def\CJKfakeboldsymbol#1{%
  \special{pdf:literal direct 2 Tr \xeCJKembold\space w}%
  \CJKoldsymbol{#1}%
  \saveCJKnode
  \special{pdf:literal direct 0 Tr}%
  \restoreCJKnode}
\def\CJKfakeboldpunctsymbol#1{%
  \special{pdf:literal direct 2 Tr \xeCJKembold\space w}%
  \CJKoldpunctsymbol{#1}%
  \saveCJKnode
  \special{pdf:literal direct 0 Tr}%
  \restoreCJKnode}
\newcommand\CJKfakebold[1]{%
  \let\CJKsymbol\CJKfakeboldsymbol
  \let\CJKpunctsymbol\CJKfakeboldpunctsymbol
  #1%
  \let\CJKsymbol\CJKoldsymbol
  \let\CJKpunctsymbol\CJKoldpunctsymbol}