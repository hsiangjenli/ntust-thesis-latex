% this file is encoded in utf-8
% v1.7
% 填入你的論文題目、姓名等資料
% 如果題目內有必須以數學模式表示的符號,請用 \mbox{} 包住數學模式,如下範例
% 如果中文名字是單名,與姓氏之間建議以全形空白填入,如下範例
% 英文名字中的稱謂,如 Prof. 以及 Dr.,其句點之後請以不斷行空白~代替一般空白,如下範例
% 如果你的指導教授沒有如預設的三位這麼多,則請把相對應的多餘教授的中文、英文名
% 的定義以空的大括號表示
% 如 \renewcommand\advisorCnameB{}
%    \renewcommand\advisorEnameB{}
%    \renewcommand\advisorCnameC{}
%    \renewcommand\advisorEnameC{}

% 論文題目 (中文)
\renewcommand\cTitle{揭開貓的秘密}
% 論文題目 (英文)
\renewcommand\eTitle{RSC: Revealing the Secret of Cats}

% 我的姓名 (中文)
\renewcommand\myCname{李貓貓}
% 我的姓名 (英文)
\renewcommand\myEname{Meow Li}
% 我的學號
\renewcommand\myStudentID{Meow12345678}

% 指導教授A的姓名 (中文)
\renewcommand\advisorCnameA{楊龍龍~博士}
% 指導教授A的姓名 (英文)
\renewcommand\advisorEnameA{Dr. Dragon Yang}

% 指導教授B的姓名 (中文)
\renewcommand\advisorCnameB{}
% 指導教授B的姓名 (英文)
\renewcommand\advisorEnameB{}

% 指導教授C的姓名 (中文)
\renewcommand\advisorCnameC{}
% 指導教授C的姓名 (英文)
\renewcommand\advisorEnameC{}

% 校名 (中文)
\renewcommand\univCname{國立臺灣科技大學}
% 校名 (英文)
\renewcommand\univEname{National Taiwan University of Science and Technology}

% 系所名 (中文)
\renewcommand\deptCname{資~訊~工~程~系}
% 系所全名 (英文)
\renewcommand\fulldeptEname{Department of Computer Science and Information Engineering}

% 系所短名 (英文, 用於書名頁學位名領域)
\renewcommand\deptEname{CSIE}
% 學院英文名 (如無,則以空的大括號表示)
\renewcommand\collEname{College Electrical Engineering and Computer Science}

% 學位名 (中文)
\renewcommand\degreeCname{碩士}
% 學位名 (英文)
\renewcommand\degreeEname{Master of Science}

% 口試年份 (中文、民國)
\renewcommand\cYear{一一二}
% 口試月份 (中文)
\renewcommand\cMonth{六} 
% 口試月份 (中文)
\renewcommand\cDay{二十六} 

% 口試年份 (阿拉伯數字、西元)
\renewcommand\eYear{2024} 
% 口試月份 (英文)
\renewcommand\eMonth{6}
% 學校所在地 (英文)
\renewcommand\ePlace{Taipei, Taiwan}

% 畢業級別;用於書背列印;若無此需要可忽略
\newcommand\GraduationClass{112}